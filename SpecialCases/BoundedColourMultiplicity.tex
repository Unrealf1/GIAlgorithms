\subsection{Графы с ограниченной кратностью цветов}

В этом алгоритме мы рассматриваем только связные графы. Если графы несвязны, достаточно попарно проверить на изоморфизм из компоненты.

\begin{definition}
    Пару $(G, f)$ назовём раскрашенным графом, если $G=(V, E)$ - это граф, а $f: V \rightarrow \mathbb{N}$ - отображение, определённое на всём $X$.
\end{definition}

\begin{definition}
    Цветом вершины $v$ назовём $f(v)$. 
\end{definition}

\begin{definition}
    Цветным классом $i \in \mathbb{N}$ назовём $f^{-1}(i)$ 
\end{definition}

\begin{definition}
    Кратность цвета i - это размер его цветного класса.
\end{definition}

Задача проверки изоморфизма раскрашенного графа отличается от определённой ранее тем, что дополнительно к условию сохранения рёбер добавляется условие сохранения цвета. То есть, если $(G=(V, E), f), (G'=(V', E'), f')$ - 2 раскршенных графа, то отображение $\phi: V \leftrightarrow V'$ должно удовлетворять двум условиям:\\
1) $(x, y) \in E \Leftrightarrow (\phi(x), \phi(y)) \in E'$\\
2) $f(x) = k \Leftrightarrow f'(\phi(x)) = k$

Приведённый ниже алгоритм решает задачу проверки двух раскрашенных графов на изоморфизм за полином, степень которого линейно зависит от максимальной кратности его цветов.

\subsubsection*{Предисловие к алгоритму}

Данный алгоритм сводит задачу проверки двух раскрашенных графов на изоморфизм к задаче нахождения системы образующих определённой группы, которая затем решается с помощью теории приведённой в предыдущем пункте.

Пусть нам даны раскрашенные графы $(G_1, f_1), (G_2, f_2)$, кратности цветов которых не превосходят константы $C$. Рассмотрим раскрашенный граф $G'=(V_1 \sqcup V_2, E_1 \sqcup E_2)$ с соответствующей раскраской. Степень кратности цветов $G'$ не превышает $2C$. Теперь рассмотрим автоморфизмы $G'$ в себя. Понятно, что исходные графы изоморфны тогда и только тогда, когда существует изоморфизм, переводящий $V_1$ в $V_2$. Чтобы проверить существование такого зоморфизма достаточно найти образующие группы автоморфизом $G'$. Алгоритм за авторством Бабаи приведённый ниже находит нужную систему образующих за полиномиальное время.

\subsubsection*{Алгоритм}

Дан граф $G$, кратности цветных классов которого не превосходят $c$, а $V = C_1 \sqcup C_2 \sqcup ... \sqcup C_k$ - разибение множества его вершин на цветные классы. Обозначим подграф $G$, порождённый объединением $C_i \sqcup C_j$ как $G_{i,j}$ (считаем, что $i < j$). Пронумеруем все такие графы и обозначим как $H_i, i \in \{1, ..., \binom{k}{2}\}$. Построим теперь последовательность графов $G_i$:
    
$G_0 = (V, \emptyset)$

$G_i = G_{i-1} \cup H_i$

Пусть теперь $A_i = Aut(G_i)$. Понятно, что $A_i \geqslant A_{i + 1}$, а значит мы получили башню групп, где $A_{\binom{k}{2}} = Aut(G)$ и $A_0 = Sym(C_1) \times ... \times Sym(C_k)$. $A_i$ полиномиально распознаваема в $A_{i-1}$, $[A_{i-1}:A_i] \leqslant (c!)^2 = const$. Система порождающих $A_0$ очевидна. Тогда согласно теореме 1 в пункте 2.2 мы можем построить систему образующих $Aut(G)$ за полиномиальное время. А это значит, что мы сможем проверить наличие нужного автоморфизма. Заметим, что $|Aut(G)| = [A_0 : A_1] [A_1 : A_2] ... [A_{\binom{k}{2}-1} : A_{\binom{k}{2}}] \leqslant (c!)^2 (\binom{k}{2}-1)$, то есть перебор тоже будет произведён за полиномиальное время.

\subsubsection*{Послесловие к алгоритму}

Задача проверки "обычного"\ изоморфизма графов сводится к данной раскрашиванием всех вершин в один цвет, но это не имеет смысла, так как в таком случае максимальная кратность цветов графа равна количеству вершин и верхняя оценка времени работы получается хуже, чем в случае тривиального алгоритма. Но данный алгоритм используется для проверки на изоморфизм некоторых других классов графов.