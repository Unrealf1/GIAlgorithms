\section{Частные случаи}

\subsection{Планарные графы}
Задача проверки двух планарных графов на изоморфизм не просто лежит в \poly, но решается за линейное время(в модели с произвольным доступом) \cite{1} и лежит в классе \L\cite{2}. Отмечу, что дерево является частным случаем планарного графа.
\begin{definition}
    Планарным называется граф, который можно изобразить на плоскости так, чтобы никакие его два ребра не пересекались.
\end{definition}
\begin{definition}
    Плоским назовём граф, изображенный на плоскости
\end{definition}
\begin{definition}
    Граф k-связен, тогда и только когда он имеет больше чем k вершин и после удаления менее чем k любых вершин граф остаётся связным. 
\end{definition}
Я приведу первый алгоритм в этой области, созданный для несколько меньшего класса графов - для планарных 3-связных графов\cite{3}, созданный Lois Weinberg. Этот алгоритм был улучшен и обобщён (Hopcroft и Tarjan \cite{6}) до алгоритма, работающего за $O(n \cdot logn)$ для \textbf{произвольного} планарного графа. Затем данный алгоритм был ускорен до линейного времени (Hopcroft and Wong \cite{1}).
Что же особенного в планарных 3-связных графах? Дело в том, что в то время, как в общем случае порядок группы автоморфизмов графа может быть равен $|V|!$, доказано\cite{4}, что размер группы автоморфизмов для планарных 3-связных графов равен $4|E|$. Это позволяет существенно сократить перебор вариантов.
\subsubsection*{Предисловие к алгоритму}
Алгоритм, описанный ниже опирается на несколько теоретических утверждений о простых планарных 3-связных графах, которые следует упомянуть до его описания:

1) Согласно \cite{5}, такой граф имеет уникальное представление на плоскости. Стоит отметить, что именно подразумевается под уникальностью. Имеется ввиду, что его грани, ограниченные рёбрами уникальны в смысле набора рёбер, которые их задают. То есть "зеркальный" граф, полученный обращением циклов граней, тоже может являться представлением исходного графа на плоскости.

2) Как было отмечено выше, в работе \cite{4} показано, что порядок группы автоморфизмов такого графа не превосходит $4|E|$

\subsubsection*{Алгоритм}
На вход алгоритм получает простой \textit{плоский} ненаправленный 3-связный граф $G=(V, E)$. В случае компьютерной программы удобно считать, что полученный граф - это не изображение на плоскости, а структура данных, где каждой вершине сопоставлен набор рёбер в порядке обхода против часовой стрелки таким образом, что изображение данного графа не имеет пересекающихся рёбер. 

Каждому автоморфизму графа мы сопоставим некоторый однозначно задающий его код, сгенерированный во время обхода графа по эйлеровому циклу. Для того, чтобы гарантировать наличие эйлерового цикла мы продублируем каждое ребро $v \rightarrow v_1, v_2$ в графе, и выдадим полученым парам ребер противоположные направления. То есть из ребра $\{a, b\}$ мы получим направленные рёбра $(a, b), (b, a)$, $a, b \in V$. В полученном ориентированном графе $G'$ входящая степень любой вершины равна её исходящей степени, а значит эйлеров цикл существует. Обойдём данный граф по следующему правилу:

    0) Все вершины графа разделим на 2 типа: \textit{старые} и \textit{новые}, изначально все вершины графа - новые. Выберем начальное ребро и направление на нём в графе или его зеркальной копии. Это $2 \cdot 2|E|$ вариантов. Первой вершиной обхода станет первая вершина ребра, согласно выбранному направлению, второй - вторая.
    
    1) Попадая в \textit{новую} вершину, отмечаем её как старую и двигаемся по ребру, имеющему минимальный от нашего угол в направлении против часовой стрелки. В случае компьютерной программы, это будет следующее после того, по которому мы пришли к вершине, ребро в наборе.

    2) Попадая в \textit{старую} вершину по ребру, противоположное направление которого ещё не было посещено, возвращаемся по противоположному ребру.

    3) Попадая в \textit{старую} вершину по ребру, обратное которому уже было пройдено, двигаемся по ребру, имеющему минимальный от нашего угол в направлении против часовой стрелки и при этом непосещённому ранее.

Во время данного обхода каждый раз встречая \textit{новую} вершину, присваиваем ей номер. То есть начальной вершине присвоен номер 1, второй - номер 2 и так далее. Попадая в очередную вершину, после присвоения номера(если оно было необходимо), добавляем в код текущего обхода номер посещённой вершины. Таким образом, для каждого обхода получим код длины $2|E| + 1$ который удобно представить ввиде вектора.

После повторения процедуры для всех $4|E|$ обходов, получим таблицу кодов размера $4|E| \times 2|E| + 1$. Затем сортируем все полученные коды в лексикографическом порядке. Сгенерировав(и отсортировав) такую матрицу для обоих графов, вопрос об изоморфизме сводится к проверке равенства первых столбцов в этих матрицах, то есть графы изоморфны тогда и только тогда, когда их лексикографически минимальные коды совпадают.

\subsubsection*{Послесловие к алгоритму}
Стоит отметить, что если графы оказались изоморфны, то обратив процедуру обхода, описанную в алгоритме можно получить нумерацию вершин в каждом графе, сответствующую изоморфизму. 

Алгоритм так же опирается на следующие теоретических утверждения:

1) Простые 3-связные планарные графы $G_1, G_2$ изоморфны тогда и только тогда, когда множества их кодов совпадают

2) Простые 3-связные планарные графы $G_1, G_2$ изоморфны тогда и только тогда, когда выполнено одно из двух условий:
        
        \quad a) Первые строки матриц их кодов(минимальные в лексикографическом смысле коды) совпадают

        \quad b) Любые 2 строки их матриц совпадают

Отмечу, что второе утверждение позволяет сократить количество вычислений в большом количестве случаев.

Кроме того, на основе данного алгоритма можно проверять на изоморфизм и ориентированные графы. Необходимые для этого модификации описанны в оригинальной статье \cite{3}.


\subsection{Теория Бабаи-Лакса}
Для нескольких алгоритмов ниже потребуются общие определения и утверждения

Пусть $G$ - группа перестановок, действующая на $\{1, ..., n\}$, а $G_m$ - это подгруппа $G$, оставляющая $\{1, ..., m\}$ на месте. Мы получили \textit{башню групп} 
$G = G_0 \geqslant G_1 \geqslant ... \geqslant G_n = id$.\\
Пусть $K_m$ - система представителей левых смежных классов $G_m$ по $G_{m+1}$. Каждый $g \in G_m$ можно однозначно представить как $g = k_m k_{m + 1} ... k_{n - 1}$, где $k_i \in K_i$. Такое представление элемента назовём \textit{каноническим}. Систему образующих $K = \cup K_i$ назовём \textit{сильной}. Из соображения $|G_m : G_{m + 1}| \leqslant n - m$ видно, что $|K| \leqslant n^2$. 

Фурст, Хопкрофт и Лакс [нужна ссылка] предложили алгоритм, строящий сильную систему образующих по произвольной системе образующих за полиномиальное время.

ТУТ ДОЛЖНО БЫТЬ ОПИСАНИЕ АЛГОРИТМА (КАСКАД)

\subsection{Графы с ограниченной кратностью цветов}

В этом алгоритме мы рассматриваем только связные графы. Если графы несвязны, достаточно попарно проверить на изоморфизм из компоненты.

\begin{definition}
    Пару $(G, f)$ назовём раскрашенным графом, если $G=(V, E)$ - это граф, а $f: V \rightarrow \mathbb{N}$ - отображение, определённое на всём $X$.
\end{definition}

\begin{definition}
    Цветом вершины $v$ назовём $f(v)$. 
\end{definition}

\begin{definition}
    Цветным классом $i \in \mathbb{N}$ назовём $f^{-1}(i)$ 
\end{definition}

\begin{definition}
    Кратность цвета i - это размер его цветного класса.
\end{definition}

Задача проверки изоморфизма раскрашенного графа отличается от определённой ранее тем, что дополнительно к условию сохранения рёбер добавляется условие созранения цвета. То есть, если $(G=(V, E), f), (G'=(V', E'), f')$ - 2 раскршенных графа, то отображение $\phi: V \leftrightarrow V'$ должно удовлетворять двум условиям:\\
1) $(x, y) \in E \Leftrightarrow (\phi(x), \phi(y)) \in E'$\\
2) $f(x) = k \Leftrightarrow f'(\phi(x)) = k$

Приведённый ниже алгоритм решает задачу проверки двух раскрашенных графов на изоморфизм за полином, степень которого линейно зависит от максимальной кратности его цветов.   Задача проверки "обычного"\ изоморфизма графов сводится к данной раскрашиванием всех вершин в один цвет, но это не имеет смысла, так как в таком случае максимальная кратность цветов графа равна количеству вершин и верхняя оценка времени работы получается хуже, чем в случае тривиального алгоритма.

\subsubsection*{Предисловие к алгоритму}

Данный алгоритм сводит задачу проверки двух раскрашенных графов на изоморфизм к задаче нахождения системы образующих определённой группы, которая затем решается с помощью теории приведённой в предыдущем пункте.

Пусть нам даны раскрашенные графы $(G_1, f_1), (G_2, f_2)$, кратности цветов которых не превосходят константы $C$. Рассмотрим раскрашенный граф $G'=(V_1 \sqcup V_2, E_1 \sqcup E_2)$ с соответствующей раскраской. Степень кратности цветов $G'$ не превышает $2C$. Теперь рассмотрим автоморфизмы $G'$ в себя. Понятно, что исходные графы изоморфны тогда и только тогда, когда существует изоморфизм, переводящий $V_1$ в $V_2$. Чтобы проверить существование такого зоморфизма достаточно найти образующие группы автоморфизом $G'$. Алгоритм за авторством Бабаи приведённый ниже находит нужную систему образующих за полиномиальное время.

\subsubsection*{Алгоритм}

Дан граф $G$, кратности цветных классов которого не превосходят $c$, а $V = C_1 \sqcup C_2 \sqcup ... \sqcup C_k$ - разибение множества его вершин на цветные классы. Обозначим подграф $G$, порождённый объединением $C_i \sqcup C_j$ как $G_{i,j}$ (считаем, что $i < j$). Пронумеруем все такие графы и обозначим как $H_i, i \in \{1, ..., \binom{k}{2}\}$. Построим теперь последовательность графов $G_i$:
    
$G_0 = (V, \emptyset)$

$G_i = G_{i-1} \cup H_i$

Обозначим за пусть теперь $A_i = Aut(G_i)$. Понятно, что $A_i \geqslant A_{i + 1}$, а значит мы получили башню групп, где $A_{\binom{k}{2}} = Aut(G)$ и $A_0 = Sym(C_1) \times ... \times Sym(C_k)$. $A_i$ полиномиально распознаваема в $A_{i-1}$, $[A_{i-1}:A_i] \leqslant (c!)^2$. Тогда мы можем построить систему образующих $Aut(G)$ за полиномиальное время и проверить наличие нужного автоморфизма.

\subsection{Графы с ограниченной кратностью собственных значений}

\begin{definition}
    Матрицей смежности графа $G=(V, E), |V|=n$ назовём матрицу $A$ размера $n \times n$, где $A[i][j] = 1$ если $(i, j) \in E$, 0 иначе.
\end{definition}
\begin{definition}
    Собственным значением графа назовём собственное значение его матрицы смежности.
\end{definition}
\begin{definition}
    Граф $G$ назовём графом кратности $m$ в смысле собственных значений, если ни одно из собственных значений графа не имеет кратности превышающей $m$.
\end{definition}
\begin{definition}
    Группой автоморфизмов графа $G=(V, E)$ назовём набор перестановок $V$, которые сохраняют рёбра.
\end{definition}

\subsubsection*{Предисловие к алгоритму}
    В данном алгоритме может потребоваться вычислять и сравнивать числа с плавающей точкой(координаты собственных векторов и собственные значения). Достаточно вычислять их с точностью до $n^c$, где $c$ - некоторая константа. Подробнее это описано в \cite{10}

    Пусть $\{e_1, ..., e_n\}$ - стандартный базис $R^n$. Сопоставим каждому базисному вектору вершину нашего графа в соответствии с нумерацией вершин в матрице смежности(i-я вершина соответствует i-му вектору). Тогда группа автоморфизмов G порождает ортогональные линейные преобразования $R^n$(перестановки базисных векторов). Получается, что группа автоморфизмов графа - это все такие матрицы перестановок $\pi$(и только они), что $\pi$ коммутирует с матрицей смежности графа ($\pi A = A \pi$).
    
    Понятно, что в общем случае размер группы перестановок G может достигать $n!$. Однако группа перестановок графа $G$ может быть представлена как замыкание множества перестановок размера не более $n^2$ \cite{9}. Это множество мы назовём порождающим.

Данный алгоритм описан в \cite{8} и опирается на следующие теоретические утверждения(если не указано иного, то утверждение доказано в \cite{8}):
\begin{lemma}
    Проблемы проверки наличия изоморфизма между графами и нахождение порождающего множества $G_1 \sqcup G_2$ полиномиально сводимы друг к другу.
\end{lemma}
Доказательство леммы выше в \cite{7}


\begin{lemma}
    Пусть $G$ - граф с матрицой смежности $A$. Матрица перестановки $\pi$ является автоморфизмом $G$ тогда и только тогда, когда собственные подпространства $A$ инвариантны относительно $\pi$ (если $x\in S_i$, то $\pi x \in S_i$)
\end{lemma}

\subsection*{Алгоритм}

\subsubsection*{Нахождение порождающего множества группы перестановок}
Данный пункт состоит из двух частей. Результатом первой части станет сведение проблемы нахождения порождающего множества группы перестановок, к проблеме нахождения группы автоморфизмов "графа с ограниченными кратностью цветов"\,, которая определена и решена в предыдущем пункте, даны представленные явно подгруппы $H_i \leqslant Sym(C_i)$, действующие на соответствующие $C_i$, и мы хотим получить  порождающее множество для $Aut(G) \cap (H_1 \times ... \times H_s)$. Затем полученная задача сводится к "башне групп", для которой существует полиномиальное решение.

Парой разделения-разложения назовём пару $((C_1, ..., C_s), (W_1, ..., W_r))$, где $V = C_1 \sqcup ... \sqcup C_s$, $W_i, W_j$ - ортоганальны и $\mathbb{R}^n = W_1 \oplus ... \oplus W_r$. Будем называть такую пару инвариантной, если все $C_i$ и $W_j$ - инвариантны относительно $Aut(G)$. Определим отношение эквивалентности $\psi_i$: $x\psi_iy \leftrightarrow $

