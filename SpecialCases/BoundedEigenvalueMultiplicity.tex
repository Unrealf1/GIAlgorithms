\subsection{Графы с ограниченной кратностью собственных значений}

\begin{definition}
    Матрицей смежности графа $G=(V, E), |V|=n$ назовём матрицу $A$ размера $n \times n$, где $A[i][j] = 1$ если $(i, j) \in E$, 0 иначе.
\end{definition}
\begin{definition}
    Собственным значением графа назовём собственное значение его матрицы смежности.
\end{definition}
\begin{definition}
    Граф $G$ назовём графом кратности $m$ в смысле собственных значений, если ни одно из собственных значений графа не имеет кратности превышающей $m$ 
\end{definition}

\subsubsection*{Предисловие к алгоритму}
Данный алгоритм описан в \cite{8}
Данный алгоритм опирается на следующие теоретические утверждения(если ммылка не указана, то утверждение доказано в \cite{8}):
\begin{lemma}
    Проблемы проверки наличия изоморфизма между графами и нахождение порождающего множества $G_1 \sqcup G_2$ полиномиально сводимы друг к другу
\end{lemma}
Доказательство в \cite{7}

\begin{lemma}
    Пусть $G$ - граф с матрицой смежности $A$. Матрица перестановки $\pi$ является автоморфизмом $G$ тогда и только тогда, когда собственные подпространства $A$ инвариантны относительно $\pi$
\end{lemma}


