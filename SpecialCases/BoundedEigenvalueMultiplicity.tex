\subsection{Графы с ограниченной кратностью собственных значений}

\begin{definition}
    Матрицей смежности графа $G=(V, E), |V|=n$ назовём матрицу $A$ размера $n \times n$, где $A[i][j] = 1$ если $(i, j) \in E$, 0 иначе.
\end{definition}
\begin{definition}
    Собственным значением графа назовём собственное значение его матрицы смежности.
\end{definition}
\begin{definition}
    Граф $G$ назовём графом кратности $m$ в смысле собственных значений, если ни одно из собственных значений графа не имеет кратности превышающей $m$.
\end{definition}
\begin{definition}
    Группой автоморфизмов графа $G=(V, E)$ назовём набор перестановок $V$, которые сохраняют рёбра.
\end{definition}

\subsubsection*{Предисловие к алгоритму}
    В данном алгоритме может потребоваться вычислять и сравнивать числа с плавающей точкой(координаты собственных векторов и собственные значения). Достаточно вычислять их с точностью до $n^c$, где $c$ - некоторая константа. Подробнее это описано в \cite{10}

    Пусть $\{e_1, ..., e_n\}$ - стандартный базис $R^n$. Сопоставим каждому базисному вектору вершину нашего графа в соответствии с нумерацией вершин в матрице смежности(i-я вершина соответствует i-му вектору). Тогда группа автоморфизмов G порождает ортогональные линейные преобразования $R^n$(перестановки базисных векторов). Получается, что группа автоморфизмов графа - это все такие матрицы перестановок $\pi$(и только они), что $\pi$ коммутирует с матрицей смежности графа ($\pi A = A \pi$).
    
    Понятно, что в общем случае размер группы перестановок G может достигать $n!$. Однако группа перестановок графа $G$ может быть представлена как замыкание множества перестановок размера не более $n^2$ \cite{9}. Это множество мы назовём порождающим.

Данный алгоритм описан в \cite{8} и опирается на следующие теоретические утверждения(если не указано иного, то утверждение доказано в \cite{8}):
\begin{lemma}
    Проблемы проверки наличия изоморфизма между графами и нахождение порождающего множества $G_1 \sqcup G_2$ полиномиально сводимы друг к другу.
\end{lemma}
Доказательство леммы выше в \cite{7}


\begin{lemma}
    Пусть $G$ - граф с матрицой смежности $A$. Матрица перестановки $\pi$ является автоморфизмом $G$ тогда и только тогда, когда собственные подпространства $A$ инвариантны относительно $\pi$ (если $x\in S_i$, то $\pi x \in S_i$)
\end{lemma}

\subsection*{Алгоритм}

\subsubsection*{Нахождение порождающего множества группы перестановок}
Данный пункт состоит из двух частей. Результатом первой части станет сведение проблемы нахождения порождающего множества группы перестановок, к проблеме нахождения группы автоморфизмов "графа с ограниченными кратностью цветов"\,, которая определена и решена в предыдущем пункте, даны представленные явно подгруппы $H_i \leqslant Sym(C_i)$, действующие на соответствующие $C_i$, и мы хотим получить  порождающее множество для $Aut(G) \cap (H_1 \times ... \times H_s)$. Затем полученная задача сводится к "башне групп", для которой существует полиномиальное решение.

Парой разделения-разложения назовём пару $((C_1, ..., C_s), (W_1, ..., W_r))$, где $V = C_1 \sqcup ... \sqcup C_s$, $W_i, W_j$ - ортоганальны и $\mathbb{R}^n = W_1 \oplus ... \oplus W_r$. Будем называть такую пару инвариантной, если все $C_i$ и $W_j$ - инвариантны относительно $Aut(G)$. Определим отношение эквивалентности $\psi_i$: $x\psi_iy \leftrightarrow $

