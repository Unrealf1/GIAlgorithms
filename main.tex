\documentclass[fleqn]{article}

\usepackage[utf8]{inputenc}
\usepackage[russian]{babel}

\usepackage[margin=1in]{geometry}
\geometry{a4paper}

\usepackage{amsmath}
\usepackage{amssymb}
\usepackage{mathtools}
\usepackage{amsthm}
\usepackage{indentfirst}

\newcommand{\poly}{$P$}
\newtheorem{definition}{Definition}

\selectlanguage{russian}

\title{Algotithms for Graph Isomorphism problem}
\author{Фёдор Букреев}
\date{\today}

\begin{document}
\maketitle
\tableofcontents

\section{Постановка задачи}
Перед формулировкой задачи необходимо ввести базовые определения:
\subsection{Основные определения}
\begin{definition}
    Простым неориентированным графом
\end{definition}

\section{Частные случаи}
\subsection{Планарный граф}
Задача проверки двух планарных графов на изоморфизм не просто лежит в \poly, но решается за линейное время \cite{1} и лежит в классе \L\cite{2}.
\begin{definition}
    Планарным называется граф, который можно изобразить на плоскости так, чтобы никакие его два ребра не пересекались.
\end{definition}
\begin{definition}
    Плоским назовём граф, изображенный на плоскости
\end{definition}
\begin{definition}
    Граф k-связен, тогда и только когда он имеет больше чем k вершин и после удаления менее чем k любых вершин граф остаётся связным. 
\end{definition}
Я приведу первый алгоритм а этой области, созданный для несколько меньшего класса графов - для планарных 3-связных графов\cite{3}, созданный Lois Weinberg.
Что же особенного в планарных 3-связных графах? Дело в том, что в то время, как в общем случае порядок группы автоморфизмов графа может быть равен $|V|!$, доказано\cite{4}, что размер группы автоморфизмов для планарных 3-связных графов равен $4|E|$. Это позволяет существенно сократить перебор вариантов.
\subsubsection*{Предисловие к алгоритму}
Алгоритм, описанный ниже опирается на несколько теоретических утверждений:

\subsubsection*{Алгоритм}
На вход алгоритм получает простой \textit{плоский} ненаправленный 3-связный граф $G=(V, E)$. В случае компьютерной программы удобно считать, что полученный граф - это не изображение на плоскости, а структура данных, где каждой вершине сопоставлен набор рёбер в порядке обхода против часовой стрелки таким образом, что изображение данного графа не имеет пересекающихся рёбер. 

Каждому автоморфизму графа мы сопоставим некоторый однозначно задающий его код, сгенерированный во время обхода графа по эйлеровому циклу. Для того, чтобы гарантировать наличие эйлерового цикла мы продублируем каждое ребро $v \rightarrow v_1, v_2$ в графе, и выдадим полученым парам ребер противоположные направления. То есть из ребра ${a, b}$ мы получим направленные рёбра $(a, b), (b, a)$, $a, b \in V$. В полученном ориентированном графе $G'$ входящая степень любой вершины равна её исходящей степени, а значит эйлеров цикл существует. 


\section{Общий алгоритм}

\begin{thebibliography}{99}
    \bibitem{1} Hopcroft, John; Wong, J. (1974), 
    "Linear time algorithm for isomorphism of planar graphs", 
    Proceedings of the Sixth Annual ACM Symposium on Theory of Computing, 
    pp. 172–184, doi:10.1145/800119.803896.
    \bibitem{2} Datta, S.; Limaye, N.; Nimbhorkar, P.; Thierauf, T.; Wagner, F. (2009), 
    "Planar graph isomorphism is in log-space", 
    2009 24th Annual IEEE Conference on Computational Complexity, 
    p. 203, arXiv:0809.2319, doi:10.1109/CCC.2009.16, ISBN 978-0-7695-3717-7.
    \bibitem{3} Lois Weinberg,
    "A Simple and Efficient Algorithm for Determining Isomorphism  of Planar Triply Connected Graphs"
    
    https://ieeexplore.ieee.org/document/1082573/
    \bibitem{4} Lois Weinberg,
    "On the Maximum Order of the Automorphism Group of a Planar Triply Connected Graph"
    
    https://epubs.siam.org/doi/abs/10.1137/0114062
    \bibitem{5}  H. Whitney, 
    "2-isomorphic graphs",  
    Amer. J. Math., 55 (1933) 245–254, 

    https://www.jstor.org/stable/2371127?seq=1
    \bibitem{6} E. Hopcroft and R.E. Tarjan. 
    "Efficient planarity testing".
    Journal of the ACM, 21, 1974.
    \bibitem{7} Rudolf MATHON (March 1979),
    "A note on the graph isomorphism counting problem"

    \bibitem{8} Babai, László; Grigoryev, D. Yu.; Mount, David M. (1982), 
    "Isomorphism of graphs with bounded eigenvalue multiplicity", 
    Proceedings of the 14th Annual ACM Symposium on Theory of Computing, 
    pp. 310–324, doi:10.1145/800070.802206, ISBN 0-89791-070-2.
    
    \bibitem{9} Merrick Furst ; John Hopcroft ; Eugene Luks,
    "Polynomial-time algorithms for permutation groups",
    Published in 21st Annual Symposium on Foundations of Computer Science (sfcs 1980)
    
    \bibitem{10} Leighton F. T.; Miller G. L.,
    "Numerical analysis of Gaussian elimination and eigenspace calculation",
    in preparation
    \bibitem{11} Семинары Ленинградского отделения института Стеклова TOFIX 
    \bibitem{} placeholder
    \bibitem{} placeholder
\end{thebibliography}

\end{document}